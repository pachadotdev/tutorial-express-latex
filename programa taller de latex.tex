\documentclass[oneside,spanish]{article}

%%%%%%%%%%%%%%%%%%%%%%%%%%%%%%%%%%%%%%%%%%%%%%%%%%%%%%%
%\begin{paquetes}
\usepackage[utf8]{inputenc}
\usepackage[spanish]{babel}
\usepackage{array}
\usepackage{float}
\usepackage{amsmath}
\usepackage{amsfonts}
\usepackage{amssymb}
\usepackage{amstext}
\usepackage{amsthm}
\usepackage{enumerate}
\usepackage{fancyhdr}
\usepackage[usenames,dvipsnames,svgnames,table]{xcolor}
\usepackage{multirow}
\usepackage{multicol}
\usepackage{tikz}
\usepackage[linkcolor=blue,colorlinks=true,urlcolor=blue]{hyperref}

\usepackage{import}
\graphicspath{{figuras/}}

\usepackage{enumerate}

\usepackage{titlesec}

\titleformat{\section}
{\normalsize\sffamily\bfseries}
{\thesection.}{.5em}{}

\titleformat{\subsection}
{\normalsize\sffamily\bfseries}
{\thesubsection.}{.5em}{}

%\end{paquetes}
%%%%%%%%%%%%%%%%%%%%%%%%%%%%%%%%%%%%%%%%%%%%%%%%%%%%%%%

%%%%%%%%%%%%%%%%%%%%%%%%%%%%%%%%%%%%%%%%%%%%%%%%%%%%%%%
%\begin{teoremas}
\theoremstyle{definition}
\newtheorem{defi}{Definición}
\newtheorem{car}{Caracterización}
\headheight 36pt
\usepackage{geometry}
\geometry{verbose,letterpaper,tmargin=30mm,bmargin=30mm,lmargin=25mm,rmargin=25mm}
\parindent 0em
\parskip 0ex
%\end{teoremas}
%%%%%%%%%%%%%%%%%%%%%%%%%%%%%%%%%%%%%%%%%%%%%%%%%%%%%%%

%%%%%%%%%%%%%%%%%%%%%%%%%%%%%%%%%%%%%%%%%%%%%%%%%%%%%%%
%\begin{encabezados}
\newtheorem{ejercicio}{Ejercicio}
\newtheorem{teorema}{Teorema}
\newtheorem{lema}{Lema}
\newtheorem{corolario}{Corolario}
\theoremstyle{definition}\newtheorem{definicion}{Definición}
\theoremstyle{definition}\newtheorem{ejemplo}{Ejemplo}
\theoremstyle{remark}\newtheorem{nota}{\textsc{Nota}}
\theoremstyle{definition}\newtheorem{proposicion}{Proposición}
\theoremstyle{definition}\newtheorem{problema}{Problema}
\newcommand{\halmos}{\hspace*{\fill}$\blacksquare$}
\newenvironment{demostracion}{\textsc{Demostraci\'{o}n.}}{\halmos}
%\end{encabezados}
%%%%%%%%%%%%%%%%%%%%%%%%%%%%%%%%%%%%%%%%%%%%%%%%%%%%%%%

%%%%%%%%%%%%%%%%%%%%%%%%%%%%%%%%%%%%%%%%%%%%%%%%%%%%%%%
%\begin{comandos}
\newcommand{\bb}[1]{\mathbb{#1}}
\newcommand{\eq}[1]{\begin{equation} #1 \end{equation}}
\newcommand{\dpr}[2]{\frac{\partial #1}{\partial #2}}
\newcommand{\R}{\mathbb{R}}
\newcommand{\K}{\mathbb{K}}
\newcommand{\N}{\mathbb{N}}
\newcommand{\emqc}[1]{#1 \in \N}
\newcommand{\ssuc}{\{x_{n_k}\}_{k\in\N}}
\newcommand{\Z}{\mathbb{Z}}
\newcommand{\Q}{\mathbb{Q}}
\newcommand{\sen}{\text{sen}}
\newcommand{\senh}{\text{senh}}
\newcommand{\adh}[1]{\text{adh}({#1})}
\newcommand{\der}[1]{\text{der}({#1})}
\newcommand{\inte}[1]{\text{int}({#1})}
\newcommand{\fr}[1]{\text{fr}({#1})}
\newcommand{\co}[1]{\text{co}({#1})}
\newenvironment{respuesta}{\smallskip \textbf{\textsf{Respuesta}} \sffamily \newline}{\hfill}
\newcommand{\re}[1]{\begin{respuesta} #1 \end{respuesta}}  
\newenvironment{solucion}{\smallskip \textbf{\textsf{Soluci\'on}} \sffamily \newline}{\hfill}
\newcommand{\so}[1]{\begin{solucion} #1 \end{solucion}}  

\renewcommand{\vec}[1]{\boldsymbol{#1}}
\renewcommand{\familydefault}{\sfdefault}

%\end{comandos}
%%%%%%%%%%%%%%%%%%%%%%%%%%%%%%%%%%%%%%%%%%%%%%%%%%%%%%%

%%%%%%%%%%%%%%%%%%%%%%%%%%%%%%%%%%%%%%%%%%%%%%%%%%%%%%%
%\begin{macros}
\makeatletter
\def\ScaleIfNeeded{%
\ifdim\Gin@nat@width>\linewidth
\linewidth
\else
\Gin@nat@width
\fi
}
\makeatother
%\end{macros}
%%%%%%%%%%%%%%%%%%%%%%%%%%%%%%%%%%%%%%%%%%%%%%%%%%%%%%%

\begin{document}

\thispagestyle{fancy}
\lhead{C\'atedras Libres}
\rhead{Taller de \color{Red}\LaTeX}

\begin{center}
\Large{\textbf{\color{RoyalBlue}Programa del Taller}}
\\\normalsize Autor: Mauricio Vargas
\end{center}

\bigskip

Para aprovechar la actividad la \emph{condición necesaria} es que lleven su notebook o netbook con latex instalado y la  \emph{condición suficiente} es que tengan ganas de aprender.   

\section{Acerca del taller}
Seguramente alguna vez han tenido que ocupar Latex en alguna tarea y han encontrado que es complicado pero manejando adecuadamente su uso nos puede ahorrar mucho tiempo. 

\smallskip

El objetivo de este taller es aprender los aspectos básicos para poder escribir documentos, informes y presentaciones que requieran utilizar fórmulas matemáticas. La ventaja de Latex no radica en su facilidad de uso, sino que si manejamos algunos comandos b\'asicos se puede escribir y editar f\'ormulas matem\'aticas con mayor rapidez que en Word, Power Point y otros programas. 

\smallskip

\section{Instalaci\'on del programa}

Latex es gratuito y se puede descargar libremente. Est\'an disponibles los siguientes enlaces.

\begin{itemize}
\item Windows: 
\\ \url{http://miktex.org}

\item Mac:
\\ \url{https://tug.org/mactex/}
\end{itemize}

\smallskip

Sin embargo, para poder utilizar Latex con facilidad es altamente recomendable utilizar un editor distinto al que el programa trae consigo. Un editor recomendable es Texmaker, que es gratuito, y se puede descargar desde \url{http://www.xm1math.net/texmaker} tanto para Windows como para Mac.

\medskip

\section{Contenidos}

\subsection{Presentación del programa}
\begin{enumerate}
\item Barras de herramientas
\item Uso del asistente
\end{enumerate}

\subsection{Composición de textos}
\begin{enumerate}
\item	Alineación del texto
\item	Tamaño y estilo del texto
\item	Saltos de línea y de página
\item	Caracteres especiales y acentuación
\item	Título, capítulos y secciones
\item	Notas al pie
\item	Números de página
\item	Palabras enfatizadas
\item	Entornos
\begin{itemize}
\item	Listas
\item	Citas
\item	Tablas
\end{itemize}
\end{enumerate}

\subsection{Composición de fórmulas matemáticas}
\begin{enumerate}
\item	Fórmulas matemáticas básicas
\begin{itemize}
\item	Entornos \$, equation y gather
\end{itemize}
\item	Ecuaciones alineadas
\begin{itemize}
\item	Entornos align y eqnarray
\end{itemize}
\item	Flechas, vectores, cuatificadores, alfabeto griego y símbolos especiales
\item	Uso de matrices
\item	Entornos
\begin{itemize}
\item	Teoremas
\item	Lemas
\item	Corolarios
\item	Ejemplos/Ejercicios
\end{itemize}
\end{enumerate}

\subsection{Formato}
\begin{enumerate}
\item	Creación de tabla de contenido y tabla de imágenes
\item	Tipografías
\item	Personalizar encabezados y pies de página
\item	Estilo de títulos y subtítulos
\item	Creación de índice alfabético (para documentos largos)
\end{enumerate}

\subsection{Gráficos}
\begin{enumerate}
\item Insertar im\'agenes en jpg, png, etc
\item Insertar gr\'aficos en pdf
\item Uso de inkscape
\end{enumerate}

\subsection{Presentaciones}
\begin{enumerate}
\item Uso de beamer
\end{enumerate}

\end{document} 