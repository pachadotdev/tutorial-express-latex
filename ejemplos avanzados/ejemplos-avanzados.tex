\documentclass[oneside,spanish]{article}

%%%%%%%%%%%%%%%%%%%%%%%%%%%%%%%%%%%%%%%%%%%%%%%%%%%%%%%
%\begin{paquetes}
\usepackage[utf8]{inputenc}
\usepackage[spanish]{babel}
\usepackage{array}
\usepackage{amsmath}
\usepackage{amsfonts}
\usepackage{amssymb}
\usepackage{amstext}
\usepackage{amsthm}
\usepackage{fancyhdr}
\usepackage{multirow}
\usepackage{multicol}
\usepackage{tikz}
\usepackage[linkcolor=blue,colorlinks=true,urlcolor=blue]{hyperref}

\usepackage{import}
\graphicspath{{figuras/}}

\usepackage{titlesec}

\titleformat{\section}
{\normalsize\sffamily\bfseries}
{\thesection.}{.5em}{}

\titleformat{\subsection}
{\normalsize\sffamily\bfseries}
{\thesubsection.}{.5em}{}

%\end{paquetes}
%%%%%%%%%%%%%%%%%%%%%%%%%%%%%%%%%%%%%%%%%%%%%%%%%%%%%%%

%%%%%%%%%%%%%%%%%%%%%%%%%%%%%%%%%%%%%%%%%%%%%%%%%%%%%%%
\theoremstyle{definition}
\newtheorem{defi}{Definición}
\newtheorem{car}{Caracterización}
\headheight 36pt
\usepackage{geometry}
\geometry{verbose,letterpaper,tmargin=30mm,bmargin=30mm,lmargin=25mm,rmargin=25mm}
\parindent 0em
\parskip 1ex
%%%%%%%%%%%%%%%%%%%%%%%%%%%%%%%%%%%%%%%%%%%%%%%%%%%%%%%

%%%%%%%%%%%%%%%%%%%%%%%%%%%%%%%%%%%%%%%%%%%%%%%%%%%%%%%
\newtheorem{ejercicio}{Ejercicio}
\newtheorem{teorema}{Teorema}
\newtheorem{lema}{Lema}
\newtheorem{corolario}{Corolario}
\theoremstyle{definition}\newtheorem{definicion}{Definición}
\theoremstyle{definition}\newtheorem{ejemplo}{Ejemplo}
\theoremstyle{remark}\newtheorem{nota}{\textsc{Nota}}
\theoremstyle{definition}\newtheorem{proposicion}{Proposición}
\theoremstyle{definition}\newtheorem{problema}{Problema}
\newcommand{\halmos}{\hspace*{\fill}$\blacksquare$}
\newenvironment{demostracion}{\textit{Demostraci\'{o}n.}}{\halmos}
%%%%%%%%%%%%%%%%%%%%%%%%%%%%%%%%%%%%%%%%%%%%%%%%%%%%%%%

%%%%%%%%%%%%%%%%%%%%%%%%%%%%%%%%%%%%%%%%%%%%%%%%%%%%%%%
\newcommand{\bb}[1]{\mathbb{#1}}
\newcommand{\eq}[1]{\begin{equation} #1 \end{equation}}
\newcommand{\dpr}[2]{\frac{\partial #1}{\partial #2}}
\newcommand{\R}{\mathbb{R}}
\newcommand{\K}{\mathbb{K}}
\newcommand{\N}{\mathbb{N}}
\newcommand{\Z}{\mathbb{Z}}
\newcommand{\Q}{\mathbb{Q}}

\newcommand{\suc}{\{x_{n}\}_{n\in\N}}
\newcommand{\sen}{\text{sen}}
\newcommand{\senh}{\text{senh}}
\newcommand{\adh}[1]{\text{adh}({#1})}
\newcommand{\der}[1]{\text{der}({#1})}
\newcommand{\inte}[1]{\text{int}({#1})}
\newcommand{\fr}[1]{\text{fr}({#1})}
\newcommand{\co}[1]{\text{co}({#1})}

\renewcommand{\vec}[1]{\boldsymbol{#1}}
%\renewcommand{\familydefault}{\sfdefault}

%%%%%%%%%%%%%%%%%%%%%%%%%%%%%%%%

\begin{document}

\begin{center}
\textsf{\textbf{Ejercicios Avanzados}}\\
Profesor: Mauricio Vargas\\
\end{center}

\thispagestyle{fancy}
\lhead{C\'atedras Libres}
\rhead{Taller de \LaTeX}

\section{F\'ormulas matem\'aticas}

\textbf{Ejercicio 1.} Escribir el sistema lagrangeano:
$$L(x,y,\lambda) = x^2+y^2 + \lambda (3 - x -y)$$
\begin{align}
\dpr{L}{x}=0 & \Longrightarrow 2x - \lambda = 0 \label{ec-1} \\
\dpr{L}{y}=0 & \Longrightarrow 2y - \lambda = 0 \label{ec-2} \\
\dpr{L}{\lambda} = 0 & \Longrightarrow 3 - x - y = 0 \label{ec-3}
\end{align}

\textbf{Ejercicio 2.} Generar enlaces a las ecuaciones anteriores a fin de obtener lo siguiente:

Restando la ecuaci\'on \eqref{ec-2} a \eqref{ec-1} se obtiene
\begin{gather}
x - y = 0 \label{ec-4} \tag{*}
\end{gather}
Si reemplazamos \eqref{ec-4} en \eqref{ec-3} resulta que $x=y=3/2$.

\textbf{Ejercicio 3.} Escribir las derivadas del sistema anterior en forma vectorial, es decir
$$\nabla L(x,y,\lambda) = 0 \Leftrightarrow 
\begin{pmatrix}
2x - \lambda \\ 2y - \lambda \\ 3 - x - y
\end{pmatrix} 
=
\begin{pmatrix}
0 \\ 0 \\0
\end{pmatrix}$$

\section{Uso de entornos}

\textbf{Ejercicio 4.} Defina un entorno llamado teorema y un entorno llamado demostraci\'on para obtener el siguiente texto:

\begin{teorema}
$A$ es linealmente dependiente si y s\'olo si existe $j=1,\ldots,n$ tal que $\vec{x}_j$ es combinaci\'on lineal de $A\setminus \{\vec{x}_j\}$.
\end{teorema}

\begin{demostracion}
$A$ es linealmente dependiente si y s\'olo si
$$\sum_{i=1}^n \alpha_n \vec{x}_n = \vec{0} \text{ con } \alpha_1,\ldots,\alpha_n \text{ no todos nulos}$$
supongamos que $\alpha_j \neq 0$ para $1\leq j \leq n$ entonces
\begin{align*}
\alpha_j \vec{x}_j &= -\sum_{\substack{i=1\\i\neq j}} \alpha_i \vec{x}_i \\
\vec{x}_j &= -\frac{1}{\alpha_j}\sum_{\substack{i=1\\i\neq j}} \alpha_i \vec{x}_i 
\end{align*}
si definimos $\beta_i = -\alpha_i / \alpha_j$ se tiene
$$\vec{x}_j = -\sum_{\substack{i=1\\i\neq j}} \beta_i \vec{x}_i $$
y por lo tanto $\vec{x}_j$ es combinaci\'on lineal de $A\setminus \{\vec{x}_j\}$.
\end{demostracion}

\textbf{Ejercicio 5.} Usando el entorno de teorema enuncie el teorema del valor medio, incluya las respectivas im\'agenes y el disclaimer que aparecen a continuaci\'on:

\begin{teorema}{\rm (Teorema del valor medio en $\R$)\index{Teorema!del valor medio en $\R$}}\label{valormedioenr}
\\Sean $[a,b]$ un intervalo cerrado y acotado y $f:[a,b]\rightarrow \R$ continua y derivable en $(a,b)$. Entonces, existe un punto $c \in (a,b)$ tal que
$$f'(c)=\frac{f(b)-f(a)}{b-a}$$
\end{teorema}

\begin{demostracion}
Buscar en \href{http://es.wikipedia.org}{Wikipedia}\footnote{Lo relevante es aprender Latex.}
\end{demostracion}

La interpretaci\'on geom\'etrica del teorema \ref{valormedioenr} es la siguiente: Si trazamos una secante que une dos puntos de una funci\'on continua y derivable, entonces
existe un punto donde la tangente al gr\'afico de la funci\'on y la secante ya definida son paralelas.
\begin{figure}[h]
	\centering
	\input{valormedio.pdf_tex}
	\caption{Teorema del valor medio.}
\end{figure}

\textbf{Ejercicio 6.} Las im\'agenes tambi\'en se incluyen dentro de un entorno. Inserte la siguiente imagen:

\begin{figure}[h]
\centering
\includegraphics[scale=0.15]{logo.jpg}
\caption{El logo de la iniciativa C\'atedras Libres}
\end{figure}

\newpage

\textbf{Ejericio 7.} El entorno verbatim permite mostrar el c\'odigo utilizado. Escriba lo siguiente:

El c\'odigo:
\begin{verbatim}
Existen bastantes colores en latex como {\color{red}rojo}, 
{\color{blue}azul}, {\color{green}verde}, etc.
\end{verbatim}

Genera lo siguiente:

Existen bastantes colores en latex como {\color{red}rojo}, {\color{blue}azul}, {\color{green}verde}, etc.

\section{Tablas}

\textbf{Ejercicio 8.} Inserte la siguiente tabla y el problema:

En \href{http://es.wikipedia.org/wiki/Programacion_lineal}{Optimizaci\'on Lineal}, la forma de pasar del problema primal al dual se puede resumir de la siguiente forma:
\begin{center}
\begin{tabular}{|c|c|}\hline
Minimizaci\'on & Maximizaci\'on \\ \hline
Restricci\'on & Variable \\
$\leq$ & $\leq$ \\
$\geq$ & $\geq$ \\
$=$ & $\in \R$ \\ \hline
Variable & Restricci\'on \\
$\leq$ & $\geq$ \\
$\geq$ & $\leq$ \\
$\in \R$ & $=$ \\ \hline
\end{tabular}
\end{center}
Es decir, el dual de un problema ser\'ia algo de la siguiente forma:
$$
\begin{array}{ccrcrcccccrcrcc}
P) & \max			& 3x_1   & +   & 2x_2  &   	     &    & \qquad & D) & \min	 		& 2y_1   & + & 4y_2    & 		 &   \\
	& \text{s.a}  & -2x_1   & +   & x_2	&\leq	 & 2 & \qquad &		& \text{s.a}  & -2y_1   & +  & 2y_2    &\geq  & 3 \\
	& 					& 2x_1 & +  & x_2     	&\geq	 & 4 & \qquad & 	 	&					& y_1 & + & y_2    &\geq  & 2 \\
	& 					& x_1   & ,   &  x_2 	    &\geq   & 0 & \qquad & 	 	&					& y_1 \geq 0  & ,  & y_2 \leq 0 	   &		& 
\end{array}
$$

\end{document} 