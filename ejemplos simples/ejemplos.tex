\documentclass[10pt,a4paper]{article}
\usepackage[utf8]{inputenc}
\usepackage[spanish]{babel}
\usepackage{amsmath}
\usepackage{amsfonts}
\usepackage{amssymb}
\usepackage[left=2cm,right=2cm,top=2cm,bottom=2cm]{geometry}
\usepackage{url}
\usepackage{color}
\usepackage[linkcolor=blue,colorlinks=true,urlcolor=blue]{hyperref}
\parindent=0mm
\parskip=1em

\begin{document}

Ejercicio 1: Escribir el cuadrado de binomio

\begin{enumerate}
\item F\'ormula junto al texto: $ (a+b)^2 = a^2 + 2ab + b^2$
\item F\'ormula centrada: Existen varias formas equivalentes (se distinguen en el c\'odigo salvo cuando est\'an numeradas)
	$$(a+b)^2 = a^2 + 2ab + b^2 $$
	\begin{equation}
	(a+b)^2 = a^2 + 2ab + b^2
	\end{equation}

	\begin{eqnarray}
	(a+b)^2 &=& a^2 + 2ab + b^2 \\
	(x+y)^2 &=& x^2 + 2xy + y^2	
	\end{eqnarray}		
\item F\'ormula con enumeraci\'on ``especial'':
	\begin{gather}\tag{Cuadrado de binomio}
	(a+b)^2 = a^2 + 2ab + b^2
	\end{gather}
\end{enumerate}

Ejercicio 2: Escribir funciones
\begin{enumerate}
\item Composici\'on de funciones: $f(x_1,x_2,x_3)=\ln(x_1^{x_2})+x_3$
\item Definir una funci\'on:
	\begin{align}
	f: \mathbb{R} &\to \mathbb{R} \cr
	(x_1,x_2)&\mapsto x_1^\alpha\cdot x_2^\beta
	\end{align}
\end{enumerate}

Ejercicio 3: Sistemas de ecuaciones
\begin{itemize}
\item Sistema de ecuaciones de $2 \times 2$:
	\begin{align*}
	x+2y&=5 \cr
	2x+3y&=4
	\end{align*}
	
\item Sistema anterior pero en forma matricial:
	\begin{gather}	
		\begin{bmatrix}
		1 & 2 \cr
		2 & 3
		\end{bmatrix}
		\begin{bmatrix}
		x \cr
		y
		\end{bmatrix}
		=
		\begin{bmatrix}
		5 \cr
		4
		\end{bmatrix}
	\end{gather}
	\begin{gather}\label{panconqueso}	
		\begin{pmatrix}
		1 & 2 \cr
		2 & 3
		\end{pmatrix}
		\begin{pmatrix}
		x \cr
		y
		\end{pmatrix}
		=
		\begin{pmatrix}
		5 \cr
		4
		\end{pmatrix}
	\end{gather}	
\end{itemize}

Ejercicio 3: Tipos de par\'entesis, sumatorias, integrales, sub\'indices y super\'indices
\begin{enumerate}
\item Sucesi\'on en $\mathbb{R}$: $\{x_i\}_{i=1}^n$
\item Sumatorias: Por ejemplo $\sum_{i\in I} \sum_{j\in J(y_0)} \sum_{k\in K_1} x_i y_j z_k$ y para que los \'indices no aparezcan hacia la derecha agregamos el comando displaystyle (ver c\'odigo) y queda $\displaystyle \sum_{i\in I} \sum_{j\in J(y_0)} \sum_{k\in K_1} x_i y_j z_k$
\item Integrales:
$$\iint_S f(x,y)dxdy$$
$$\int_1^5 \int_ 0^3 xy dxdy = \int_1^5\bigg( \int_0^3 xy dx\bigg)dy$$
\end{enumerate}

Ejercicio 4: Espacios y cuantificadores.\newline
El logar\'itmo natural es una funci\'on bien definida para todo $x$ estrictamente positivo. Es decir,
$$\ln : \mathbb{R}_+ \to \mathbb{R}$$
y es la inversa de la exponencial por lo que $\exp(\ln(x))=x,~\forall x>0$. Claramente $\nexists x \leq 0$ tal que la imagen de $x$ est\'e definida pero $\exists x > 0$, de hecho cualquier valor estrictamente positivo, para el cual la imagen de $x$ est\'a definida.

\section*{Teorema de Farkas}
\textbf{Teorema 1.} \textsf{(Farkas, 1897)} 
\textit{Uno y s\'olo uno de los siguientes sistemas tiene soluci\'on:}
\begin{eqnarray}
A\vec{x}=\vec{b}&,& \vec{x} \geq \vec{0} \label{farkas1} \\
A^T\vec{y} \geq \vec{0}&,&\vec{b}^T \vec{y} <0 \label{farkas2}
\end{eqnarray}

\textsc{Demostraci\'on}. Buscar en Wikipedia {\color{blue} \url{http://es.wikipedia.org}}. 

En general las ecuaciones \eqref{farkas1} y \eqref{farkas2} tienen varias implicancias: Permiten determinar condiciones de \'optimo de un problema generalizado y no necesariamente lineal (\url{http://en.wikipedia.org/wiki/Nonlinear_programming})

\end{document}
