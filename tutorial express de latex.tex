\documentclass[letterpaper,twoside]{article}

%%%%%%%%%%%%%%%%%%%%%%%%%%%%%%%%%%%%%%%%%%%%%%%%%%%%%%%

\usepackage[spanish]{babel}
\usepackage{fancyhdr}

%para manipular imagenes
\usepackage{float}
\usepackage{tikz}

%para tener todos los simbolos
\usepackage{amsmath}
\usepackage{amssymb}
\usepackage{amsthm}

\usepackage{color}
\usepackage[linkcolor=blue,colorlinks=true,urlcolor=blue]{hyperref}

%sangria e interlineado
\parskip=1em
\parindent=0em

%margenes
\headheight 33pt
\topmargin 0.2cm
\oddsidemargin 1cm
\evensidemargin 1cm
\textwidth 15cm
\textheight 19cm

%separador de miles en espaniol
\spanishdecimal{.}


\title{Tutorial Express de \LaTeX}
\author{Mauricio Vargas}
\makeindex

\pagestyle{fancy}
\lhead{Tutorial Express de Latex}
\rhead{\href{pachamaltese.github.io}{Mauricio Vargas S.}}

\begin{document}

\maketitle
\tableofcontents

\section{Motivaci\'on}

\LaTeX\: no es muy amistoso para trabajar con tablas e im\'agenes. Siendo justos debemos decir que las ventajas de Latex se centran en la rapidez y facilidad para escribir f\'ormulas matem\'aticas sumado a que es muy f\'acil editarlas. 

A diferencia de programas como Word, donde es f\'acil poner im\'agenes pero es complicado escribir y editar f\'ormulas, Latex no est\'a pensado para ser un programa de uso general sino que se enfoca a la creaci\'on de informes, art\'iculos y libros cient\'ificos. Por esto es que es un programa de especial utilidad para estudiantes y profesionales del \'area de ciencias e ingenier\'ia como tambi\'en para quienes necesiten utilizar herramientas estad\'isticas.

Muchas revistas, entre estas la \href{http://www.ams.org/home/page}{American Mathematical Society Journal}, tienen disponibles en sus p\'aginas varias plantillas para enviarles art\'iculos, con la finalidad de que los autores s\'olo se preocupen de escribir y no de los aspectos est\'eticos de los documentos. Algunas editoriales como \href{http://www.ams.org/home/page}{Springer} utilizan Latex pues simplifica enormemente la escritura de textos largos y que contengan f\'ormulas matem\'aticas.

Daremos algunos ejemplos para complementar lo visto en los talleres realizados para la iniciativa \href{https://www.facebook.com/catedraslibres?fref=ts}{C\'atedras Libres}.

Este material est\'a disponible bajo la \href{http://creativecommons.org/licenses/by-nc/3.0/}{Licencia Creative Commons}. Se puede redistribuir y editar libremente, sin usos comerciales, citando debidamente la fuente.

\newpage

\section{?`C\'omo insertar ecuaciones simples?}

El cuadrado de binomio\footnote{\quad \url{http://es.wikipedia.org/wiki/Cuadrado_de_un_binomio}}, de ecuaci\'on $(a+b)^2$, se desarrolla como sigue 
$$(a+b)^2 = (a+b)(a+b) = a^2 + 2ab + b^2$$

De todas maneras conviene numerar ecuaciones notando que esta ecuaci\'on es independiente de las variables empleadas, es decir,
\begin{equation}
(a+b)^2 = a^2 + 2ab + b^2
\end{equation}
es lo mismo que
\begin{eqnarray}
(x+y)^2 &=& x^2+2xy+y^2 \\
(p+q)^2 &=& p^2+2pq+q^2
\end{eqnarray}

\newpage

\section{?`C\'omo insertar ecuaciones simples con letras griegas y s\'imbolos especiales?}

Podemos usar numeraciones especiales para las ecuaciones o incluso dejarlas con sus respectivos nombres

\textbf{Serie geom\'etrica}: Sea $r\in (0,1)$ se tiene que la suma de exponentes consecutivos de $r^i$ converge, es decir
\begin{gather}
\sum_{i=0}^{\infty} r^i = \frac{1}{1-r} \tag{serie geom\'etrica}
\end{gather}
notamos que $r$ puede crecer de manera tal que
$$
\lim_{r\to \infty} \frac{1}{1-r} = 0
$$
o escrito de otra manera
$$
\sum_{i=0}^{\infty} r^i = \frac{1}{1-r} \stackrel{r\to \infty}{\longrightarrow} 0
$$

Por otra parte puede que nos interese escribir en caracteres un poco m\'as complejos. Daremos dos ejemplos:

\textbf{Distribuci\'on Normal}
\begin{equation}
\mathcal{N}(\mu,\sigma^2) = \int\limits_{-\infty}^{x} \frac1{\sigma\sqrt{2\pi}}\: \exp{-\frac{1}{2}\left(\frac{t-\mu}{\sigma}\right)^2}\: dt
\end{equation}

\textbf{Lagrangeano de un sistema}
\begin{equation}\label{lagrangeano}
L(\vec{x},\vec{\lambda},\vec{\mu}) = f(\vec{x}) + \sum_{i\in I} \lambda_i g_i(\vec{x}) + \sum_{j\in J(\vec{x}_0)} \mu_j h_j(\vec{x})
\end{equation}

\newpage

\section{Insertar Cuantificadores y operadores matem\'aticos}

\subsection{Uni\'on e intersecci\'on}

Supongamos que $A_1$ y $A_2$ son disjuntos, entonces
$$
(A_1 \cup A_2 \neq \varnothing) \wedge (A_2 \cap A_2 = \varnothing)
$$

Si tenemos $n$ conjuntos $A_i$, consideremos el conjunto $\Lambda = \{1,2,\ldots , n\}$, entonces la uni\'on e intersecci\'on de $n$ partes se escribe
$$
\bigcup_{i\in \Lambda} A_i
$$

$$
\bigcap_{i\in \Lambda} A_i
$$

\subsection{Uso de cuantificadores}

\textbf{Continuidad de una funci\'on de una variable}

Una funci\'on $f$ es \emph{continua} en un punto $x_0$ en el dominio de la funci\'on
si $\forall \varepsilon > 0 \: \exists \delta> 0$ tal que para todo $x$ en el dominio de la funci\'on se tiene que:
$$|x-x_0|<\delta \Rightarrow |f(x)-f(x_0)|<\varepsilon$$

\newpage

\section{Ejemplos con matrices}

Tenemos dos estilos de matrices
$$
\begin{pmatrix}
1 & 0 \\
0 & 1
\end{pmatrix}
\text{ o tambi\'en }
\begin{bmatrix}
1 & 0 \\
0 & 1
\end{bmatrix}
$$

La inversa de la matriz\footnote{Se pueden importar matrices desde Matlab, las de este ejemplo se obtuvieron de esta forma}
$$
\begin{pmatrix}
2 & 1 & 2 \\
0 & 3 & 1 \\
5 & 6 & 7
\end{pmatrix}
$$
es la matriz
$$
\begin{pmatrix}
    3.0000  &  1.0000  &  -1.0000 \\
    1.0000  &  0.8000  & -0.4000 \\
   -3.0000  &  -1.4000 &  1.2000
\end{pmatrix}
$$
entonces
$$
\begin{pmatrix}
2 & 1 & 2 \\
0 & 3 & 1 \\
5 & 6 & 7
\end{pmatrix}
\begin{pmatrix}
    3.0000  &  1.0000  &  -1.0000 \\
    1.0000  &  0.8000  & -0.4000 \\
   -3.0000  &  -1.4000 &  1.2000
\end{pmatrix}
=
\begin{pmatrix}
1 & 0 & 0 \\
0 & 1 & 0 \\
0 & 0 & 1
\end{pmatrix}
$$

\newpage

\section[Tablas]{Tablas\footnote{\quad Este t\'opico es bastante complicado de dominar en Latex por lo que se ver\'a en detalles (combinando celdas, etc) en un taller avanzado}}

Tabla b\'asica:

\begin{center}
\begin{tabular}{|c|c|}
\hline 
\text{Rango et\'areo} & \text{Frecuencia} \\ 
\hline 
[0-15] & 20 \\ 
\hline 
[15-25] & 10 \\ 
\hline 
\end{tabular}
\end{center}

\newpage

\section{Consideraciones adicionales con algunos s\'imbolos}

Los signos \%, \&, etc, como realizan distintas tareas tales como hacer que una l\'inea sea ignorada en el pdf o alinear matrices, se deben escribir anteponiendo $\backslash$.

De no hacerlo tendr\'iamos que una frase como \emph{``Las acciones subieron un 2''} en lugar de \emph{``Las acciones subieron un 2\%''}.

Los acentos y e\~nes se pueden escribir directamente, sin embargo es recomendable escribirlos de la siguiente forma:

\begin{verbatim}
\'a \'e \'i \'o \'u \~n
\end{verbatim}

para obtener
\begin{center}
\'a \'e \'i \'o \'u \~n
\end{center}

\newpage

\section{Insertar figuras}

\begin{figure}[h]
\centering
\includegraphics[scale=1]{Flag_of_Hong_Kong.png} 
\caption{\textsf{Bandera de Hong Kong.}}
\end{figure}

\begin{figure}[h]
\centering
\includegraphics[scale=1]{Flag_of_Hong_Kong_(1959-1997).png} 
\caption{\textsf{Antigua bandera de Hong Kong.}}
\end{figure}


\newpage

\section{Insertar enlaces y material \'util}

%Si nos interesa saber m\'as de la ecuaci\'on \eqref{lagrangeano} se puede buscar en Wikipedia:\\
%\url{http://es.wikipedia.org/wiki/Condiciones_de_Karush-Kuhn-Tucker}

Para complementar lo visto en el taller, cuya idea era aprender lo b\'asico de Latex, existe un buen manual para usuarios intermedios y avanzados:\\ 
\url{http://es.wikibooks.org/wiki/Manual_de_LaTeX}

Tambi\'en es posible exportar tablas desde excel y ahorrarse cabezazos (y muchas l\'ineas de c\'odigo) con un complemento para excel creado por la comunidad de usuarios de Latex:\\
\url{http://www.ctan.org/tex-archive/support/excel2latex} 

\end{document}